\documentclass[10pt,a4paper]{article}
\usepackage[utf8]{inputenc}
\usepackage{amsmath}
\usepackage{dsfont}
\usepackage{amsfonts}
\usepackage{amssymb}

\author{Leander van Beek}
\title{Highly divisible triangular number}

\newcommand{\desda}{\textbf{if and only if} }

\begin{document}

\maketitle

\textbf{Problem:} 

The sequence of triangle numbers is generated by adding the natural numbers. So the 7th triangle number would be 1 + 2 + 3 + 4 + 5 + 6 + 7 = 28. The first ten terms would be:

1, 3, 6, 10, 15, 21, 28, 36, 45, 55, ...

Let us list the factors of the first seven triangle numbers:

     1: 1
     3: 1,3
     6: 1,2,3,6
    10: 1,2,5,10
    15: 1,3,5,15
    21: 1,3,7,21
    28: 1,2,4,7,14,28

We can see that 28 is the first triangle number to have over five divisors.

What is the value of the first triangle number to have over five hundred divisors?


\vspace{0.5cm}
\hrule
\vspace{0.5cm}

\section{Solution}



\end{document}